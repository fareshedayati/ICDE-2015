\section{Introduction}
% no \IEEEPARstart

High dimensional supervised learning problems such as text or image annotation are common in many research and practical problems. Such problems require defining a mapping between the raw input space and a target vector space, where each dimension is called a feature to some categorical or numerical output variable.
 A sample is a input-output pair. While those datasets have very high dimensional input space,
they are often sparsely representable. For instance, the number of unique words
associated to a text document is actually small compared to the number of words of a
given language. In order to work efficiently with such data, efficient matrix
formats have been developed with fast operations, such as dot product, and
low memory footprints.

Tree-based ensemble models, such as adaboost \cite{freund1995desicion}, random
forest \cite{breiman2001random} or gradient tree boosting
\cite{friedman2001greedy}, are some of the most robust and widely-used
supervised machine learning. What all these methods have in common is that they use
randomized decision trees as a base learner. This building block is a
hierarchical model which divide the input space through a series of binary
splitting rules which partition the input space. Predictions of a decision tree is obtained by
following the tree structure until reaching a leaf. In the ensemble framework, those models are either averaged
\cite{breiman2001random} or learnt sequentially \cite{freund1995desicion},
\cite{friedman2001greedy}.

Many models, such as linear or nearest neighbors model, could directly benefit
from the input sparsity by formulating the entire algorithm through a set of
dot products. However this is not possible for tree based methods, most machine
learning packages don't support sparse input for tree-based methods, are
restricted to decision stumps (decision tree with only one internal node) or
have a sub-optimal implementation through the simulation of a random access
memory as in the dense case. The only solution is often to densify the input
space which leads first to severe memory constraints and then to slow
training time.

In this paper, we present an efficient splitting procedure tailored for
numerical sparse input data in compressed sparse column format, a sparse matrix
format. For a given subset of samples, we are able to efficiently extract non
zero values for a given feature of this subset of samples. Knowing which
elements are nonzero allows large speedup. It decreases sorting time of
samples in the current node along features which is an essential component in
all tree-based models. Moreover it reduces the set of possible splits to
evaluate at each node. We also want to highlight that the contribution of this
paper have been proposed for and merged in the  \emph{scikit-learn}
\cite{buitinck2013api,pedregosa2011scikit} open source package. This will
benefit the machine learning community.

The rest of this paper is organized as follows: Section \ref{sec:background}
introduces decision tree splitting algorithm and sparse matrix formats; Section
\ref{sec:sparse-input-dt} describes the proposed splitting algorithm for sparse
input data; Section \ref{sec:experiments} provides our empirical implementation
study and Section \ref{sec:conclusion} concludes and describes further
perspectives.

